% Part II

    \section{\texorpdfstring{}{\thesection}}


\largdb{Discretized one-dimensional differential equation of $u(x)$ with boundary values as follows}

\begin{center}
    $\left\{\begin{tabular}{l}
        $-(a(x)u'(x))'+\lambda u(x)=x-\frac{1}{2}, \ x \in [0,1]$ \\
        $u(0)=u(1)=0$ \\
    \end{tabular}\right.$
\end{center}


\largdb{\noindent by centered difference scheme with $n$ interior mesh points. In $(2)$, $a(x)$ is a given function
and $\lambda$ is a constant, taking the following choices:}

\begin{tabular}{@{\hspace{0ex}}p{20em}}
    $(1)~a(x)=2, \lambda=100,10,1,-1.$ \\
    $(2)~a(x)=x, \lambda=1,10,100,1000.$
\end{tabular}

\largdb{Solve the resulting linear systems by the steepest descent method (search and study this
algorithm if you do not know it) and conjugate gradient method. Explain the problem
and procedures that you perform, and report and comment on the results you observed.
If the methods do not work for certain choices of $a(x)$ and $\lambda$, what would you like to do
to solve the problem.}

\largdb{Note that you need to test your code with different size ($n = 100; 200; 400; 800; 1600; 3200$
or larger if your computer permits) for both parts and comment on the results you obtain.}

\el

{\LARGE \textbf{Problem and Procedures:}}

first, we use the steepest descent method, which uses the opposite of the Gradient path to reach to the minimum of $(Ax-b)$.

This Algorithm is presented here:

There are many optimizations done on this code, like
\begin{itemize}
    \item saving A*d once and using it later.
    \item saving $d^T* d$ once and using it later.
    \item calculating ${d^T* (A * d)}$ instead of ${(d^T* A) * d}$ which is very faster,
    \item ...
\end{itemize}

    \begin{algorithm}[H]
        \begin{algorithmic}[1]
            \caption{Steepest Descent Method}
            \State $\db{Input}(A,b,maxtry,tol)$
            \State $d=b$
            \State $x=0$
            \For{$tries=1..maxtry$}
                \State $\alpha=\frac{d^T* d}{d^T* A * d}$
                \State $x=x+\alpha * d$
                \Comment {Actually, every 100 steps, I calculate d this way: $d=b-A*x$}
                \State $d=d-alpha*A*d$
                \If{$(d^T* d)/n < tol$}
                    \State \db{return}
                \EndIf
            \EndFor
            \State \db{return}
        \end{algorithmic}
    \end{algorithm}

Then, the conjugate gradient method Algorithm:

The optimizations  also are applied here
    \begin{algorithm}[H]
        \begin{algorithmic}[1]
            \caption{Conjugate Gradient Method}
            \State $\db{Input}(A,b,maxtry,tol)$
            \State $d=b$
            \State $p=d$
            \State $x=0$
            \For{$tries=1..maxtry$}
                \State $\alpha=\frac{d^T* d}{p^T* A * p}$
                \State $x=x+\alpha * p$
                \Comment {Actually, every 100 steps, I calculate d this way: $d=b-A*x$}
                \State $D=d^T*d$
                \State $d=d-alpha*A*p$
                \State $\beta=d^T*d/D$
                \State $p=r+\beta*p$
                \If{$(d^T* d)/n < tol$}
                    \State \db{return}
                \EndIf
            \EndFor
            \State \db{return}
        \end{algorithmic}
    \end{algorithm}


    
{\LARGE \textbf{Solution Details:}}

In the steepest descent method, as stated above, the convergence criterion is: $(d^T* d)/n < tol$

In the conjugate gradient method, as stated above, the convergence criterion is the same: $(d^T* d)/n < tol$

The maximum number of iterations for both is 100,000

The tolerance is $10^{-7}$

  
\clearpage
{\LARGE \textbf{Results:}}

Here we can see the results for different cases:

%------------------------------------------------------------------------------
%------------------------------------------------------------------------------
%       STEEPEST DECENT METHOD
%------------------------------------------------------------------------------
%------------------------------------------------------------------------------

{\Large \textbf{Steepest Decent Method (u versus x)}}

The results for Case 1, $a(x)=2$ can be seen in Figure \ref{fig:fig2SD1}.

\begin{figure}[H]
    \centering
    $\begin{array}{cc}
        \includegraphics[width=3.4in]{figure2_SD1} &
        \includegraphics[width=3.4in]{figure2_SD2} \\
        \includegraphics[width=3.4in]{figure2_SD3} &
        \includegraphics[width=3.4in]{figure2_SD4} \\
    \end{array}$
    \caption{Steepest Decent Method, $a(x)=2$}
    \label{fig:fig2SD1}
\end{figure}

We can see in the figure above that the steepest descent method has not converged properly for large matrices, because it takes a lot of steps to converge and the maximum is 100,000. that is because the solution path will be zig-zag with very small improvement at each step. The problem is reahing the minimum of (Ax-b) which is ill-conditioned as will be explained later and the surface shape is a very wide ellipse.

\clearpage
The results for Case 2, $a(x)=x$ can be seen in Figure \ref{fig:fig2SD5}.

\begin{figure}[H]
    \centering
    $\begin{array}{cc}
        \includegraphics[width=3.4in]{figure2_SD5} &
        \includegraphics[width=3.4in]{figure2_SD6} \\
        \includegraphics[width=3.4in]{figure2_SD7} &
        \includegraphics[width=3.4in]{figure2_SD8} \\
    \end{array}$
    \caption{Steepest Decent Method, $a(x)=x$}
    \label{fig:fig2SD5}
\end{figure}

In case 2, only for $\lambda=100$, we see good solution. That is because of very low condition number in that case. The worst cases are always the ones with lowest $\lambda$ which make the biggest condition numbers.

As we can see in the figures \ref{fig:fig2SD1} and \ref{fig:fig2SD5}, some cases still have not converged after 100,000 steps.


%------------------------------------------------------------------------------
%------------------------------------------------------------------------------
%       Conjugate Gradient Method
%------------------------------------------------------------------------------
%------------------------------------------------------------------------------
\clearpage
{\Large \textbf{Conjugate Gradient Method (u versus x)}}

The results for Case 1, $a(x)=2$ can be seen in Figure \ref{fig:fig2CG1}.

\begin{figure}[H]
    \centering
    $\begin{array}{cc}
        \includegraphics[width=3.4in]{figure2_CG1} &
        \includegraphics[width=3.4in]{figure2_CG2} \\
        \includegraphics[width=3.4in]{figure2_CG3} &
        \includegraphics[width=3.4in]{figure2_CG4} \\
    \end{array}$
    \caption{Conjugate Gradient Method, $a(x)=2$}
    \label{fig:fig2CG1}
\end{figure}

Here, we see that the CG method is very stronger and converges well in case 1.

\clearpage
The results for Case 2, $a(x)=x$ can be seen in Figure \ref{fig:fig2CG5}.

\begin{figure}[H]
    \centering
    $\begin{array}{cc}
        \includegraphics[width=3.4in]{figure2_CG5} &
        \includegraphics[width=3.4in]{figure2_CG6} \\
        \includegraphics[width=3.4in]{figure2_CG7} &
        \includegraphics[width=3.4in]{figure2_CG8} \\
    \end{array}$
    \caption{Conjugate Gradient Method, $a(x)=x$}
    \label{fig:fig2CG5}
\end{figure}

As we can see in figure \ref{fig:fig2CG5}, the cases for $a(x)=x$ with $\lambda=1$ and $\lambda=10$ have convergence problems.

\clearpage
{\Large \textbf{The condition numbers}}

The condition numbers are very important, because they determine the speed of convergence and accuracy.

\begin{figure}[H]
    \centering
    $\begin{array}{cc}
        \includegraphics[width=3.4in]{figure2_cond11.eps} 
        \includegraphics[width=3.4in]{figure2_cond12.eps} 
    \end{array}$        
    \caption{Condition numbers versus size, Case 1 and 2 }
    \label{fig:fig2CG5}
\end{figure}

{\LARGE \textbf{Observations \slash~  Explanations:}}

\begin{itemize}

\item Some cases do not converge in 100,000 steps and are very ill-conditioned: 

\item case 1: $\lambda=-1$, $\lambda=1$, and even to some extent, $\lambda=10$, are ill-conditioned. As we can see in Figure \ref{fig:fig2CG5}, the condition number for these 3 cases are all greater than $10^{+6}$, which is very high. The steepest descent method has problems in this case and does not converge in big matrices

\item case 2: $\lambda=1$ and to some extent $\lambda=10$ have high condition numbers according to Figure \ref{fig:fig2CG5}. These are the worst cases and both methods fail at giving the proper answers for big matrices.

\item The steepest descent method is very slow in all cases. In general, the steepest descent method with $cond(A)>10^5$ has convergence problems and in big matrices, it will not converge in 100,000 steps, even with small tolerance =$10^{-3}$

\item The conjugate gradient method has problems for cases with $cond(A)>3*10^{6}$.

\item Below, we can see a plot of the convergence iterations for both methods.


\begin{figure}[H]
    \centering
    $\begin{array}{cc}
        \includegraphics[width=3.4in]{figure2_tries21.eps} 
        \includegraphics[width=3.4in]{figure2_tries22.eps} 
    \end{array}$        
    \caption{Steepest Descent Method, Number of Iterations,  Cases 1 and 2 }
    \label{fig:fig2SDtries}
\end{figure}

\begin{figure}[H]
    \centering
    $\begin{array}{cc}
        \includegraphics[width=3.4in]{figure2_triesCG13.eps} 
        \includegraphics[width=3.4in]{figure2_triesCG14.eps} 
    \end{array}$        
    \caption{Conjugate Gradient Method, Number of Iterations,  Cases 1 and 2 }
    \label{fig:fig2CGtries}
\end{figure}

\item Tolerance is $10^{-5}$ for CG and $10^{-3}$ for SD. In Figures \ref{fig:fig2SDtries} and \ref{fig:fig2CGtries}, we observe the number iterations performed. Max number of iterations is 100,000.

\item As we can see, the conjugate gradient does well on case 1, but will increase the number of iterations a lot when $\lambda=1,10$. That is because of high condition number.
\item For CG, I suggest using a pre-conditioner for case 2, $\lambda=1,10$. we could also use more number of steps.
\item. The reason for slow convergence is that the problem is ill-conditioned and the gradient is not directing to the peak. in CG, even when we are correcting our path, we still get the zig-zag convergence. This path goes back-and forth when approaching the answer and causes slow convergence in both cases. The gradient direction causes the most problems for the steepest descent method.
    
\item in the following figures, we see the convergence process of CG and SD for case 2.

\begin{figure}[H]
    \centering
    \includegraphics[width=3.4in]{figure2_convCG1.eps} 
    \caption{Conjugate gradient method, Convergence trend,  case 2, $\lambda=1,100$}
    \label{fig:fig2CGconv1}
\end{figure}

\item In this figure, we observe that because of high condition number for $\lambda=1$, the convergence is very slower. A pre-conditioner would help a lot

\begin{figure}[H]
    \centering
    $\begin{array}{cc}
        \includegraphics[width=3.4in]{figure2_convSD1.eps} 
        \includegraphics[width=3.4in]{figure2_convSD2.eps} 
    \end{array}$        
    \caption{Steepest Descent Method, Number of Iterations,  Case 2, $\lambda=1,1000$ }
    \label{fig:fig2CGtries}
\end{figure}

\item In case 2, $\lambda=1$, the convergence process is very bad, as can be seen in the figure above. Again, that is because of the high condition number. using a preconditioner will help a lot.


\end{itemize}
